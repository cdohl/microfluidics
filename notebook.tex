
% Default to the notebook output style

    


% Inherit from the specified cell style.




    
\documentclass[11pt]{article}

    
    
    \usepackage[T1]{fontenc}
    % Nicer default font (+ math font) than Computer Modern for most use cases
    \usepackage{mathpazo}

    % Basic figure setup, for now with no caption control since it's done
    % automatically by Pandoc (which extracts ![](path) syntax from Markdown).
    \usepackage{graphicx}
    % We will generate all images so they have a width \maxwidth. This means
    % that they will get their normal width if they fit onto the page, but
    % are scaled down if they would overflow the margins.
    \makeatletter
    \def\maxwidth{\ifdim\Gin@nat@width>\linewidth\linewidth
    \else\Gin@nat@width\fi}
    \makeatother
    \let\Oldincludegraphics\includegraphics
    % Set max figure width to be 80% of text width, for now hardcoded.
    \renewcommand{\includegraphics}[1]{\Oldincludegraphics[width=.8\maxwidth]{#1}}
    % Ensure that by default, figures have no caption (until we provide a
    % proper Figure object with a Caption API and a way to capture that
    % in the conversion process - todo).
    \usepackage{caption}
    \DeclareCaptionLabelFormat{nolabel}{}
    \captionsetup{labelformat=nolabel}

    \usepackage{adjustbox} % Used to constrain images to a maximum size 
    \usepackage{xcolor} % Allow colors to be defined
    \usepackage{enumerate} % Needed for markdown enumerations to work
    \usepackage{geometry} % Used to adjust the document margins
    \usepackage{amsmath} % Equations
    \usepackage{amssymb} % Equations
    \usepackage{textcomp} % defines textquotesingle
    % Hack from http://tex.stackexchange.com/a/47451/13684:
    \AtBeginDocument{%
        \def\PYZsq{\textquotesingle}% Upright quotes in Pygmentized code
    }
    \usepackage{upquote} % Upright quotes for verbatim code
    \usepackage{eurosym} % defines \euro
    \usepackage[mathletters]{ucs} % Extended unicode (utf-8) support
    \usepackage[utf8x]{inputenc} % Allow utf-8 characters in the tex document
    \usepackage{fancyvrb} % verbatim replacement that allows latex
    \usepackage{grffile} % extends the file name processing of package graphics 
                         % to support a larger range 
    % The hyperref package gives us a pdf with properly built
    % internal navigation ('pdf bookmarks' for the table of contents,
    % internal cross-reference links, web links for URLs, etc.)
    \usepackage{hyperref}
    \usepackage{longtable} % longtable support required by pandoc >1.10
    \usepackage{booktabs}  % table support for pandoc > 1.12.2
    \usepackage[inline]{enumitem} % IRkernel/repr support (it uses the enumerate* environment)
    \usepackage[normalem]{ulem} % ulem is needed to support strikethroughs (\sout)
                                % normalem makes italics be italics, not underlines
    

    
    
    % Colors for the hyperref package
    \definecolor{urlcolor}{rgb}{0,.145,.698}
    \definecolor{linkcolor}{rgb}{.71,0.21,0.01}
    \definecolor{citecolor}{rgb}{.12,.54,.11}

    % ANSI colors
    \definecolor{ansi-black}{HTML}{3E424D}
    \definecolor{ansi-black-intense}{HTML}{282C36}
    \definecolor{ansi-red}{HTML}{E75C58}
    \definecolor{ansi-red-intense}{HTML}{B22B31}
    \definecolor{ansi-green}{HTML}{00A250}
    \definecolor{ansi-green-intense}{HTML}{007427}
    \definecolor{ansi-yellow}{HTML}{DDB62B}
    \definecolor{ansi-yellow-intense}{HTML}{B27D12}
    \definecolor{ansi-blue}{HTML}{208FFB}
    \definecolor{ansi-blue-intense}{HTML}{0065CA}
    \definecolor{ansi-magenta}{HTML}{D160C4}
    \definecolor{ansi-magenta-intense}{HTML}{A03196}
    \definecolor{ansi-cyan}{HTML}{60C6C8}
    \definecolor{ansi-cyan-intense}{HTML}{258F8F}
    \definecolor{ansi-white}{HTML}{C5C1B4}
    \definecolor{ansi-white-intense}{HTML}{A1A6B2}

    % commands and environments needed by pandoc snippets
    % extracted from the output of `pandoc -s`
    \providecommand{\tightlist}{%
      \setlength{\itemsep}{0pt}\setlength{\parskip}{0pt}}
    \DefineVerbatimEnvironment{Highlighting}{Verbatim}{commandchars=\\\{\}}
    % Add ',fontsize=\small' for more characters per line
    \newenvironment{Shaded}{}{}
    \newcommand{\KeywordTok}[1]{\textcolor[rgb]{0.00,0.44,0.13}{\textbf{{#1}}}}
    \newcommand{\DataTypeTok}[1]{\textcolor[rgb]{0.56,0.13,0.00}{{#1}}}
    \newcommand{\DecValTok}[1]{\textcolor[rgb]{0.25,0.63,0.44}{{#1}}}
    \newcommand{\BaseNTok}[1]{\textcolor[rgb]{0.25,0.63,0.44}{{#1}}}
    \newcommand{\FloatTok}[1]{\textcolor[rgb]{0.25,0.63,0.44}{{#1}}}
    \newcommand{\CharTok}[1]{\textcolor[rgb]{0.25,0.44,0.63}{{#1}}}
    \newcommand{\StringTok}[1]{\textcolor[rgb]{0.25,0.44,0.63}{{#1}}}
    \newcommand{\CommentTok}[1]{\textcolor[rgb]{0.38,0.63,0.69}{\textit{{#1}}}}
    \newcommand{\OtherTok}[1]{\textcolor[rgb]{0.00,0.44,0.13}{{#1}}}
    \newcommand{\AlertTok}[1]{\textcolor[rgb]{1.00,0.00,0.00}{\textbf{{#1}}}}
    \newcommand{\FunctionTok}[1]{\textcolor[rgb]{0.02,0.16,0.49}{{#1}}}
    \newcommand{\RegionMarkerTok}[1]{{#1}}
    \newcommand{\ErrorTok}[1]{\textcolor[rgb]{1.00,0.00,0.00}{\textbf{{#1}}}}
    \newcommand{\NormalTok}[1]{{#1}}
    
    % Additional commands for more recent versions of Pandoc
    \newcommand{\ConstantTok}[1]{\textcolor[rgb]{0.53,0.00,0.00}{{#1}}}
    \newcommand{\SpecialCharTok}[1]{\textcolor[rgb]{0.25,0.44,0.63}{{#1}}}
    \newcommand{\VerbatimStringTok}[1]{\textcolor[rgb]{0.25,0.44,0.63}{{#1}}}
    \newcommand{\SpecialStringTok}[1]{\textcolor[rgb]{0.73,0.40,0.53}{{#1}}}
    \newcommand{\ImportTok}[1]{{#1}}
    \newcommand{\DocumentationTok}[1]{\textcolor[rgb]{0.73,0.13,0.13}{\textit{{#1}}}}
    \newcommand{\AnnotationTok}[1]{\textcolor[rgb]{0.38,0.63,0.69}{\textbf{\textit{{#1}}}}}
    \newcommand{\CommentVarTok}[1]{\textcolor[rgb]{0.38,0.63,0.69}{\textbf{\textit{{#1}}}}}
    \newcommand{\VariableTok}[1]{\textcolor[rgb]{0.10,0.09,0.49}{{#1}}}
    \newcommand{\ControlFlowTok}[1]{\textcolor[rgb]{0.00,0.44,0.13}{\textbf{{#1}}}}
    \newcommand{\OperatorTok}[1]{\textcolor[rgb]{0.40,0.40,0.40}{{#1}}}
    \newcommand{\BuiltInTok}[1]{{#1}}
    \newcommand{\ExtensionTok}[1]{{#1}}
    \newcommand{\PreprocessorTok}[1]{\textcolor[rgb]{0.74,0.48,0.00}{{#1}}}
    \newcommand{\AttributeTok}[1]{\textcolor[rgb]{0.49,0.56,0.16}{{#1}}}
    \newcommand{\InformationTok}[1]{\textcolor[rgb]{0.38,0.63,0.69}{\textbf{\textit{{#1}}}}}
    \newcommand{\WarningTok}[1]{\textcolor[rgb]{0.38,0.63,0.69}{\textbf{\textit{{#1}}}}}
    
    
    % Define a nice break command that doesn't care if a line doesn't already
    % exist.
    \def\br{\hspace*{\fill} \\* }
    % Math Jax compatability definitions
    \def\gt{>}
    \def\lt{<}
    % Document parameters
    \title{00 Course Outline SS2020}
    
    
    

    % Pygments definitions
    
\makeatletter
\def\PY@reset{\let\PY@it=\relax \let\PY@bf=\relax%
    \let\PY@ul=\relax \let\PY@tc=\relax%
    \let\PY@bc=\relax \let\PY@ff=\relax}
\def\PY@tok#1{\csname PY@tok@#1\endcsname}
\def\PY@toks#1+{\ifx\relax#1\empty\else%
    \PY@tok{#1}\expandafter\PY@toks\fi}
\def\PY@do#1{\PY@bc{\PY@tc{\PY@ul{%
    \PY@it{\PY@bf{\PY@ff{#1}}}}}}}
\def\PY#1#2{\PY@reset\PY@toks#1+\relax+\PY@do{#2}}

\expandafter\def\csname PY@tok@gd\endcsname{\def\PY@tc##1{\textcolor[rgb]{0.63,0.00,0.00}{##1}}}
\expandafter\def\csname PY@tok@gu\endcsname{\let\PY@bf=\textbf\def\PY@tc##1{\textcolor[rgb]{0.50,0.00,0.50}{##1}}}
\expandafter\def\csname PY@tok@gt\endcsname{\def\PY@tc##1{\textcolor[rgb]{0.00,0.27,0.87}{##1}}}
\expandafter\def\csname PY@tok@gs\endcsname{\let\PY@bf=\textbf}
\expandafter\def\csname PY@tok@gr\endcsname{\def\PY@tc##1{\textcolor[rgb]{1.00,0.00,0.00}{##1}}}
\expandafter\def\csname PY@tok@cm\endcsname{\let\PY@it=\textit\def\PY@tc##1{\textcolor[rgb]{0.25,0.50,0.50}{##1}}}
\expandafter\def\csname PY@tok@vg\endcsname{\def\PY@tc##1{\textcolor[rgb]{0.10,0.09,0.49}{##1}}}
\expandafter\def\csname PY@tok@vi\endcsname{\def\PY@tc##1{\textcolor[rgb]{0.10,0.09,0.49}{##1}}}
\expandafter\def\csname PY@tok@vm\endcsname{\def\PY@tc##1{\textcolor[rgb]{0.10,0.09,0.49}{##1}}}
\expandafter\def\csname PY@tok@mh\endcsname{\def\PY@tc##1{\textcolor[rgb]{0.40,0.40,0.40}{##1}}}
\expandafter\def\csname PY@tok@cs\endcsname{\let\PY@it=\textit\def\PY@tc##1{\textcolor[rgb]{0.25,0.50,0.50}{##1}}}
\expandafter\def\csname PY@tok@ge\endcsname{\let\PY@it=\textit}
\expandafter\def\csname PY@tok@vc\endcsname{\def\PY@tc##1{\textcolor[rgb]{0.10,0.09,0.49}{##1}}}
\expandafter\def\csname PY@tok@il\endcsname{\def\PY@tc##1{\textcolor[rgb]{0.40,0.40,0.40}{##1}}}
\expandafter\def\csname PY@tok@go\endcsname{\def\PY@tc##1{\textcolor[rgb]{0.53,0.53,0.53}{##1}}}
\expandafter\def\csname PY@tok@cp\endcsname{\def\PY@tc##1{\textcolor[rgb]{0.74,0.48,0.00}{##1}}}
\expandafter\def\csname PY@tok@gi\endcsname{\def\PY@tc##1{\textcolor[rgb]{0.00,0.63,0.00}{##1}}}
\expandafter\def\csname PY@tok@gh\endcsname{\let\PY@bf=\textbf\def\PY@tc##1{\textcolor[rgb]{0.00,0.00,0.50}{##1}}}
\expandafter\def\csname PY@tok@ni\endcsname{\let\PY@bf=\textbf\def\PY@tc##1{\textcolor[rgb]{0.60,0.60,0.60}{##1}}}
\expandafter\def\csname PY@tok@nl\endcsname{\def\PY@tc##1{\textcolor[rgb]{0.63,0.63,0.00}{##1}}}
\expandafter\def\csname PY@tok@nn\endcsname{\let\PY@bf=\textbf\def\PY@tc##1{\textcolor[rgb]{0.00,0.00,1.00}{##1}}}
\expandafter\def\csname PY@tok@no\endcsname{\def\PY@tc##1{\textcolor[rgb]{0.53,0.00,0.00}{##1}}}
\expandafter\def\csname PY@tok@na\endcsname{\def\PY@tc##1{\textcolor[rgb]{0.49,0.56,0.16}{##1}}}
\expandafter\def\csname PY@tok@nb\endcsname{\def\PY@tc##1{\textcolor[rgb]{0.00,0.50,0.00}{##1}}}
\expandafter\def\csname PY@tok@nc\endcsname{\let\PY@bf=\textbf\def\PY@tc##1{\textcolor[rgb]{0.00,0.00,1.00}{##1}}}
\expandafter\def\csname PY@tok@nd\endcsname{\def\PY@tc##1{\textcolor[rgb]{0.67,0.13,1.00}{##1}}}
\expandafter\def\csname PY@tok@ne\endcsname{\let\PY@bf=\textbf\def\PY@tc##1{\textcolor[rgb]{0.82,0.25,0.23}{##1}}}
\expandafter\def\csname PY@tok@nf\endcsname{\def\PY@tc##1{\textcolor[rgb]{0.00,0.00,1.00}{##1}}}
\expandafter\def\csname PY@tok@si\endcsname{\let\PY@bf=\textbf\def\PY@tc##1{\textcolor[rgb]{0.73,0.40,0.53}{##1}}}
\expandafter\def\csname PY@tok@s2\endcsname{\def\PY@tc##1{\textcolor[rgb]{0.73,0.13,0.13}{##1}}}
\expandafter\def\csname PY@tok@nt\endcsname{\let\PY@bf=\textbf\def\PY@tc##1{\textcolor[rgb]{0.00,0.50,0.00}{##1}}}
\expandafter\def\csname PY@tok@nv\endcsname{\def\PY@tc##1{\textcolor[rgb]{0.10,0.09,0.49}{##1}}}
\expandafter\def\csname PY@tok@s1\endcsname{\def\PY@tc##1{\textcolor[rgb]{0.73,0.13,0.13}{##1}}}
\expandafter\def\csname PY@tok@dl\endcsname{\def\PY@tc##1{\textcolor[rgb]{0.73,0.13,0.13}{##1}}}
\expandafter\def\csname PY@tok@ch\endcsname{\let\PY@it=\textit\def\PY@tc##1{\textcolor[rgb]{0.25,0.50,0.50}{##1}}}
\expandafter\def\csname PY@tok@m\endcsname{\def\PY@tc##1{\textcolor[rgb]{0.40,0.40,0.40}{##1}}}
\expandafter\def\csname PY@tok@gp\endcsname{\let\PY@bf=\textbf\def\PY@tc##1{\textcolor[rgb]{0.00,0.00,0.50}{##1}}}
\expandafter\def\csname PY@tok@sh\endcsname{\def\PY@tc##1{\textcolor[rgb]{0.73,0.13,0.13}{##1}}}
\expandafter\def\csname PY@tok@ow\endcsname{\let\PY@bf=\textbf\def\PY@tc##1{\textcolor[rgb]{0.67,0.13,1.00}{##1}}}
\expandafter\def\csname PY@tok@sx\endcsname{\def\PY@tc##1{\textcolor[rgb]{0.00,0.50,0.00}{##1}}}
\expandafter\def\csname PY@tok@bp\endcsname{\def\PY@tc##1{\textcolor[rgb]{0.00,0.50,0.00}{##1}}}
\expandafter\def\csname PY@tok@c1\endcsname{\let\PY@it=\textit\def\PY@tc##1{\textcolor[rgb]{0.25,0.50,0.50}{##1}}}
\expandafter\def\csname PY@tok@fm\endcsname{\def\PY@tc##1{\textcolor[rgb]{0.00,0.00,1.00}{##1}}}
\expandafter\def\csname PY@tok@o\endcsname{\def\PY@tc##1{\textcolor[rgb]{0.40,0.40,0.40}{##1}}}
\expandafter\def\csname PY@tok@kc\endcsname{\let\PY@bf=\textbf\def\PY@tc##1{\textcolor[rgb]{0.00,0.50,0.00}{##1}}}
\expandafter\def\csname PY@tok@c\endcsname{\let\PY@it=\textit\def\PY@tc##1{\textcolor[rgb]{0.25,0.50,0.50}{##1}}}
\expandafter\def\csname PY@tok@mf\endcsname{\def\PY@tc##1{\textcolor[rgb]{0.40,0.40,0.40}{##1}}}
\expandafter\def\csname PY@tok@err\endcsname{\def\PY@bc##1{\setlength{\fboxsep}{0pt}\fcolorbox[rgb]{1.00,0.00,0.00}{1,1,1}{\strut ##1}}}
\expandafter\def\csname PY@tok@mb\endcsname{\def\PY@tc##1{\textcolor[rgb]{0.40,0.40,0.40}{##1}}}
\expandafter\def\csname PY@tok@ss\endcsname{\def\PY@tc##1{\textcolor[rgb]{0.10,0.09,0.49}{##1}}}
\expandafter\def\csname PY@tok@sr\endcsname{\def\PY@tc##1{\textcolor[rgb]{0.73,0.40,0.53}{##1}}}
\expandafter\def\csname PY@tok@mo\endcsname{\def\PY@tc##1{\textcolor[rgb]{0.40,0.40,0.40}{##1}}}
\expandafter\def\csname PY@tok@kd\endcsname{\let\PY@bf=\textbf\def\PY@tc##1{\textcolor[rgb]{0.00,0.50,0.00}{##1}}}
\expandafter\def\csname PY@tok@mi\endcsname{\def\PY@tc##1{\textcolor[rgb]{0.40,0.40,0.40}{##1}}}
\expandafter\def\csname PY@tok@kn\endcsname{\let\PY@bf=\textbf\def\PY@tc##1{\textcolor[rgb]{0.00,0.50,0.00}{##1}}}
\expandafter\def\csname PY@tok@cpf\endcsname{\let\PY@it=\textit\def\PY@tc##1{\textcolor[rgb]{0.25,0.50,0.50}{##1}}}
\expandafter\def\csname PY@tok@kr\endcsname{\let\PY@bf=\textbf\def\PY@tc##1{\textcolor[rgb]{0.00,0.50,0.00}{##1}}}
\expandafter\def\csname PY@tok@s\endcsname{\def\PY@tc##1{\textcolor[rgb]{0.73,0.13,0.13}{##1}}}
\expandafter\def\csname PY@tok@kp\endcsname{\def\PY@tc##1{\textcolor[rgb]{0.00,0.50,0.00}{##1}}}
\expandafter\def\csname PY@tok@w\endcsname{\def\PY@tc##1{\textcolor[rgb]{0.73,0.73,0.73}{##1}}}
\expandafter\def\csname PY@tok@kt\endcsname{\def\PY@tc##1{\textcolor[rgb]{0.69,0.00,0.25}{##1}}}
\expandafter\def\csname PY@tok@sc\endcsname{\def\PY@tc##1{\textcolor[rgb]{0.73,0.13,0.13}{##1}}}
\expandafter\def\csname PY@tok@sb\endcsname{\def\PY@tc##1{\textcolor[rgb]{0.73,0.13,0.13}{##1}}}
\expandafter\def\csname PY@tok@sa\endcsname{\def\PY@tc##1{\textcolor[rgb]{0.73,0.13,0.13}{##1}}}
\expandafter\def\csname PY@tok@k\endcsname{\let\PY@bf=\textbf\def\PY@tc##1{\textcolor[rgb]{0.00,0.50,0.00}{##1}}}
\expandafter\def\csname PY@tok@se\endcsname{\let\PY@bf=\textbf\def\PY@tc##1{\textcolor[rgb]{0.73,0.40,0.13}{##1}}}
\expandafter\def\csname PY@tok@sd\endcsname{\let\PY@it=\textit\def\PY@tc##1{\textcolor[rgb]{0.73,0.13,0.13}{##1}}}

\def\PYZbs{\char`\\}
\def\PYZus{\char`\_}
\def\PYZob{\char`\{}
\def\PYZcb{\char`\}}
\def\PYZca{\char`\^}
\def\PYZam{\char`\&}
\def\PYZlt{\char`\<}
\def\PYZgt{\char`\>}
\def\PYZsh{\char`\#}
\def\PYZpc{\char`\%}
\def\PYZdl{\char`\$}
\def\PYZhy{\char`\-}
\def\PYZsq{\char`\'}
\def\PYZdq{\char`\"}
\def\PYZti{\char`\~}
% for compatibility with earlier versions
\def\PYZat{@}
\def\PYZlb{[}
\def\PYZrb{]}
\makeatother


    % Exact colors from NB
    \definecolor{incolor}{rgb}{0.0, 0.0, 0.5}
    \definecolor{outcolor}{rgb}{0.545, 0.0, 0.0}



    
    % Prevent overflowing lines due to hard-to-break entities
    \sloppy 
    % Setup hyperref package
    \hypersetup{
      breaklinks=true,  % so long urls are correctly broken across lines
      colorlinks=true,
      urlcolor=urlcolor,
      linkcolor=linkcolor,
      citecolor=citecolor,
      }
    % Slightly bigger margins than the latex defaults
    
    \geometry{verbose,tmargin=1in,bmargin=1in,lmargin=1in,rmargin=1in}
    
    

    \begin{document}
    
    
    \maketitle
    
    

    
    \section{Course Outline}\label{course-outline}

Welcome to the new semester, this time fully online. We start with an
introduction of what fluids (gases and liquids) are and set the
foundation for the first part of the course on \emph{microfluidics}.

\subsection{Week 1: General Introduction
(7.4.2020)}\label{week-1-general-introduction-7.4.2020}

We will cover the adminstration and give a introduction into fluid
mechanics and microfluidics in particular.

Gases and liquids can be modeled with particles (non interacting and
interacting). Molecular dynamics (MD) simulations help to understand
their behaviour and watch them in slow motion. To get into the mood of
using averaged equations I want you to play with a MD program. It can be
\href{http://cav2012.sg/cdohl/ph3501/md/}{run in your browser}. For more
information on how to run the program is available
\href{http://physics.weber.edu/schroeder/md/}{web site from Prof
Schroeder}, please note that this version does not have the pressure
graph.

To get us all started, we rehearse some math. Please look and work
through the
\href{01\%20Math\%20Refresher\%20and\%20Python\%20Introduction.ipynb}{Math
refresher notebook}. This notebook also offers you with some first
introduction into Python, the language we use in the course. A nice
\href{http://nbviewer.jupyter.org/github/jrjohansson/scientific-python-lectures/blob/master/Lecture-1-Introduction-to-Python-Programming.ipynb}{introduction
is given by Robert Johansson}. Don't worry, we will slowly start with
it. I'll motivate the use of a programming language in the course, and
besides learning fluid mechanics you obtain some skills which increases
your employability!

We have a short recap on the *
\href{01\%20Math\%20Refresher\%20and\%20Python\%20Introduction.ipynb}{math}
and the representation in Python then we'll look into the *
\href{02\%20Streamlines,\%20Pathlines,\%20Streaklines.ipynb}{streamlines,
streaklines, and pathlines} and use a
\href{Example\%20of\%20Stream-Streak-Pathlines.ipynb}{interactive
example} before we look at the

\subsection{Week 2: Intro to Microfluidics \& Derivation of the
Conservation of
Mass}\label{week-2-intro-to-microfluidics-derivation-of-the-conservation-of-mass}

Short introduction to microfluidics plus 3 videos are available on
youtube here: \href{https://www.youtube.com/watch?v=b8zE2i755-k}{Video
1}, \href{https://www.youtube.com/watch?v=68p3qAm4i7U}{Video 2}, and
\href{https://www.youtube.com/watch?v=EYuyRUjnTgc}{Video 3}.

Soft mathematical introduction to fluid mechanics. We start with the
\href{Material\%20Derivative,\%20Gaussian\%20Divergence\%20Theorem\%20and\%20Conservation\%20of\%20Mass.ipynb}{material
derivative} and also introduce the \emph{Gaussian Divergence Theorem}
and the \emph{Conservation of Mass}. It is important that you understand
the
\href{https://en.wikipedia.org/wiki/Lagrangian_and_Eulerian_specification_of_the_flow_field}{Eulerian
description} of flows.

\subsubsection{Tutorial 1 (Week 2):}\label{tutorial-1-week-2}

See Tutorial Sheet on E-Learning

\subsection{Week 3: Navier Stokes equation derivation and first
analytical
solutions}\label{week-3-navier-stokes-equation-derivation-and-first-analytical-solutions}

\href{Cauchy's\%20Equation\%20of\%20Motion\%20\&\%20Navier\%20Stokes\%20Equation.ipynb}{In
this Notebook} we first derive the Cauchy's Equation of Motion for a
continuum from Newton's 2nd Law of Motion for a deformable object. Then
we insert the surface stresses for a linear fluid and through
simplification obtain the Navier Stokes Equation for a incompressible
and constant viscosity fluid. Then we look at some solution of a
\href{13a_Navier\%20Stokes\%20example.ipynb}{numerical flow solver}.

\subsection{Week 4:}\label{week-4}

We discuss analytical solutions to the Navier Stokes Equation in
\href{Analytical\%20solutions\%20to\%20steady\%20and\%20laminar\%20flows.ipynb}{planar
and cylindrical symmetry} and
\href{Dimensional\%20Analysis\%20of\%20the\%20Navier\%20Stokes\%20Equation.ipynb}{Dimensional
Analysis}.

\subsubsection{Tutorial 2 (Week 4):}\label{tutorial-2-week-4}

\textbf{Question A Flow Simulations}

\begin{enumerate}
\def\labelenumi{\arabic{enumi}.}
\tightlist
\item
  Write down the boundary conditions for the velocity \(\vec{u}\) and
  pressure for a flow in a 2d-tube driven by a pressure gradient. You
  need one for each boundary, i.e. the walls and the inlet/outlet, and
  component of the velocity. The pressure at the walls are determined by
  the incompressibility condition. For help look into the code.
\item
  Do the same for a flow in a 2-tube driven by the upper wall. Again,
  check you answer with the boundary conditions stated in the code.
\item
  Conduct simulations for pressure driven and wall driven flows. Try to
  obtain a parabolic profile for the pressure driven flow and a linear
  profile for the wall driven flow. Discuss the results. What is the
  effect of the time step of integration, and the CFL-number. Look up
  the meaning of the CFL number.
\end{enumerate}

** Question B Index Notation**

Use the index notation and work out the following problems

\begin{enumerate}
\def\labelenumi{\arabic{enumi}.}
\tightlist
\item
  \[\frac{\partial}{\partial x_i}(p\,\delta_{ij})=(\nabla p)_j\]
\item
  \[\nabla\cdot(\rho \vec{u})=(\nabla \rho)\cdot \vec{u}+\rho\,\nabla\cdot\vec{u}\]
\end{enumerate}

** Question C Navier Stokes Equation**

The electrostatic force is a body force \(\vec{F}=q\,\vec{E}\), where
\(q\) is the charge and \(\vec{E}\) is the electric field. Add this
force in the derivation of the Navier Stokes Equation and write a Navier
Stokes Eq. with electrostatic body forces using a charge density
\(\rho_{el}\) and the electric field \(\vec{E}\).

\subsection{Week 5}\label{week-5}

Navier Stokes solution to an
\href{Analytical\%20Solution\%20to\%20an\%20unsteady\%20flow.ipynb}{unsteady
problem} (Stoke's first problem) and comparison with the CFD solver. For
inviscid flows we simplify the Navier Stokes Eq. to the
\href{Bernoulli\%20Equation.ipynb}{Bernoulli equation}.

\subsection{Week6}\label{week6}

Application of the
\href{Example\%20of\%20unsteady\%20Bernoulli\%20equation.ipynb}{Unsteady
Bernoulli Equation} for the collapse of a spherical void. Introduction
to
\href{Ideal\%20Flow\%20\&\%20Solutions\%20to\%20the\%20Laplace\%20Equation.ipynb}{potential
flow}.

\subsection{Tutorial 3 (Week 6)}\label{tutorial-3-week-6}

\subsubsection{Question 1}\label{question-1}

Repeat the Navier Stokes derivation for a tube now for 2 concentric
tubes with radii \(r_i\) and \(r_o\).\\
 Show that the flow velocity in \(z\)-direction is \[
u_z(r)=\frac{1}{4\mu}\frac{\mathrm{d}p}{\mathrm{d}z}\left(r^2-r_o^2+\frac{r_i^2-r_o^2}{\ln \frac{r_o}{r_i}}\ln\left[\frac{r}{r_i}\right]\right)\quad .
\]

\subsubsection{Question 2}\label{question-2}

We'll read and discuss two microfluidic papers to get a feeling on the
making and running of a microfluidic system. The two papers are
available here *
\href{pdf/manufacture\%20microfluidic\%20venturi.pdf}{Cavitation in flow
through a micro-orifice inside a silicon microchannel} from Chandan
Mishra, and Yoav Peles. This paper describes the manufacturing process
of glass bonded Si chips for high-pressure water flows. *
\href{pdf/microfluidic\%20venturi.pdf}{An experimental investigation of
hydrodynamic cavitation in micro-Venturis} from Chandan Mishra, and Yoav
Peles. This paper describes a venturi and the phase transition of the
liquid into vapor.

Work in two groups. The first group focuses reading and understanding on
manufacturing. Try to answer the questions: 1. Explain the use of a
photoresist. 2. What is the difference of SiO and pure Si surface to RIE
and DRIE processes? 3. Where is anodic bonding used. 4. How is the
device mounted in the holder and connected to the tubing?

The second group looks at the experimental paper (cavitation in
micro-Venturis). 1. The cavitation number and the Reynolds number are
defined. Check their values for the experimental protocols. Is it a low
or high-Reynolds number flow? What dominates, inerta or viscosity? 2.
Why does a low cavitation number result in more vapor generation? 3.
Summarize the regimes of cavitation observed in the experiment.

\subsection{Week 7}\label{week-7}

Derivation of the force on a sphere \href{Potential\%20Sphere.ipynb}{in
potential flow} and in a viscous flow, i.e. the
\href{Stokes\%20Sphere.ipynb}{Stokes Sphere} problem. We will also look
at a viscous microfluidic flow with a potential flow solution, the
\href{Hele\%20Shaw\%20Flow.ipynb}{Hele Shaw flow}.

\subsection{Tutorial 4 (Week 8)}\label{tutorial-4-week-8}

\section{Course Outline}\label{course-outline-1}

Welcome to the new semester, this time fully online. We start with an
introduction of what fluids (gases and liquids) are and set the
foundation for the first part of the course on \emph{microfluidics}.

\subsection{Week 1: General Introduction
(7.4.2020)}\label{week-1-general-introduction-7.4.2020-1}

We will cover the adminstration and give a introduction into fluid
mechanics and microfluidics in particular.

Gases and liquids can be modeled with particles (non interacting and
interacting). Molecular dynamics (MD) simulations help to understand
their behaviour and watch them in slow motion. To get into the mood of
using averaged equations I want you to play with a MD program. It can be
\href{http://cav2012.sg/cdohl/ph3501/md/}{run in your browser}. For more
information on how to run the program is available
\href{http://physics.weber.edu/schroeder/md/}{web site from Prof
Schroeder}, please note that this version does not have the pressure
graph.

To get us all started, we rehearse some math. Please look and work
through the
\href{01\%20Math\%20Refresher\%20and\%20Python\%20Introduction.ipynb}{Math
refresher notebook}. This notebook also offers you with some first
introduction into Python, the language we use in the course. A nice
\href{http://nbviewer.jupyter.org/github/jrjohansson/scientific-python-lectures/blob/master/Lecture-1-Introduction-to-Python-Programming.ipynb}{introduction
is given by Robert Johansson}. Don't worry, we will slowly start with
it. I'll motivate the use of a programming language in the course, and
besides learning fluid mechanics you obtain some skills which increases
your employability!

We have a short recap on the *
\href{01\%20Math\%20Refresher\%20and\%20Python\%20Introduction.ipynb}{math}
and the representation in Python then we'll look into the *
\href{02\%20Streamlines,\%20Pathlines,\%20Streaklines.ipynb}{streamlines,
streaklines, and pathlines} and use a
\href{Example\%20of\%20Stream-Streak-Pathlines.ipynb}{interactive
example} before we look at the

\subsection{Week 2: Intro to Microfluidics \& Derivation of the
Conservation of
Mass}\label{week-2-intro-to-microfluidics-derivation-of-the-conservation-of-mass-1}

Short introduction to microfluidics plus 3 videos are available on
youtube here: \href{https://www.youtube.com/watch?v=b8zE2i755-k}{Video
1}, \href{https://www.youtube.com/watch?v=68p3qAm4i7U}{Video 2}, and
\href{https://www.youtube.com/watch?v=EYuyRUjnTgc}{Video 3}.

Soft mathematical introduction to fluid mechanics. We start with the
\href{Material\%20Derivative,\%20Gaussian\%20Divergence\%20Theorem\%20and\%20Conservation\%20of\%20Mass.ipynb}{material
derivative} and also introduce the \emph{Gaussian Divergence Theorem}
and the \emph{Conservation of Mass}. It is important that you understand
the
\href{https://en.wikipedia.org/wiki/Lagrangian_and_Eulerian_specification_of_the_flow_field}{Eulerian
description} of flows.

\subsubsection{Tutorial 1 (Week 2):}\label{tutorial-1-week-2-1}

See Tutorial Sheet on E-Learning

\subsection{Week 3: Navier Stokes equation derivation and first
analytical
solutions}\label{week-3-navier-stokes-equation-derivation-and-first-analytical-solutions-1}

\href{Cauchy's\%20Equation\%20of\%20Motion\%20\&\%20Navier\%20Stokes\%20Equation.ipynb}{In
this Notebook} we first derive the Cauchy's Equation of Motion for a
continuum from Newton's 2nd Law of Motion for a deformable object. Then
we insert the surface stresses for a linear fluid and through
simplification obtain the Navier Stokes Equation for a incompressible
and constant viscosity fluid. Then we look at some solution of a
\href{13a_Navier\%20Stokes\%20example.ipynb}{numerical flow solver}.

\subsection{Week 4:}\label{week-4-1}

We discuss analytical solutions to the Navier Stokes Equation in
\href{Analytical\%20solutions\%20to\%20steady\%20and\%20laminar\%20flows.ipynb}{planar
and cylindrical symmetry} and
\href{Dimensional\%20Analysis\%20of\%20the\%20Navier\%20Stokes\%20Equation.ipynb}{Dimensional
Analysis}.

\subsubsection{Tutorial 2 (Week 4):}\label{tutorial-2-week-4-1}

\textbf{Question A Flow Simulations}

\begin{enumerate}
\def\labelenumi{\arabic{enumi}.}
\tightlist
\item
  Write down the boundary conditions for the velocity \(\vec{u}\) and
  pressure for a flow in a 2d-tube driven by a pressure gradient. You
  need one for each boundary, i.e. the walls and the inlet/outlet, and
  component of the velocity. The pressure at the walls are determined by
  the incompressibility condition. For help look into the code.
\item
  Do the same for a flow in a 2-tube driven by the upper wall. Again,
  check you answer with the boundary conditions stated in the code.
\item
  Conduct simulations for pressure driven and wall driven flows. Try to
  obtain a parabolic profile for the pressure driven flow and a linear
  profile for the wall driven flow. Discuss the results. What is the
  effect of the time step of integration, and the CFL-number. Look up
  the meaning of the CFL number.
\end{enumerate}

** Question B Index Notation**

Use the index notation and work out the following problems

\begin{enumerate}
\def\labelenumi{\arabic{enumi}.}
\tightlist
\item
  \[\frac{\partial}{\partial x_i}(p\,\delta_{ij})=(\nabla p)_j\]
\item
  \[\nabla\cdot(\rho \vec{u})=(\nabla \rho)\cdot \vec{u}+\rho\,\nabla\cdot\vec{u}\]
\end{enumerate}

** Question C Navier Stokes Equation**

The electrostatic force is a body force \(\vec{F}=q\,\vec{E}\), where
\(q\) is the charge and \(\vec{E}\) is the electric field. Add this
force in the derivation of the Navier Stokes Equation and write a Navier
Stokes Eq. with electrostatic body forces using a charge density
\(\rho_{el}\) and the electric field \(\vec{E}\).

\subsection{Week 5}\label{week-5-1}

Navier Stokes solution to an
\href{Analytical\%20Solution\%20to\%20an\%20unsteady\%20flow.ipynb}{unsteady
problem} (Stoke's first problem) and comparison with the CFD solver. For
inviscid flows we simplify the Navier Stokes Eq. to the
\href{Bernoulli\%20Equation.ipynb}{Bernoulli equation}.

\subsection{Week6}\label{week6-1}

Application of the
\href{Example\%20of\%20unsteady\%20Bernoulli\%20equation.ipynb}{Unsteady
Bernoulli Equation} for the collapse of a spherical void. Introduction
to
\href{Ideal\%20Flow\%20\&\%20Solutions\%20to\%20the\%20Laplace\%20Equation.ipynb}{potential
flow}.

\subsection{Tutorial 3 (Week 6)}\label{tutorial-3-week-6-1}

\subsubsection{Question 1}\label{question-1-1}

Repeat the Navier Stokes derivation for a tube now for 2 concentric
tubes with radii \(r_i\) and \(r_o\).\\
 Show that the flow velocity in \(z\)-direction is \[
u_z(r)=\frac{1}{4\mu}\frac{\mathrm{d}p}{\mathrm{d}z}\left(r^2-r_o^2+\frac{r_i^2-r_o^2}{\ln \frac{r_o}{r_i}}\ln\left[\frac{r}{r_i}\right]\right)\quad .
\]

\subsubsection{Question 2}\label{question-2-1}

We'll read and discuss two microfluidic papers to get a feeling on the
making and running of a microfluidic system. The two papers are
available here *
\href{pdf/manufacture\%20microfluidic\%20venturi.pdf}{Cavitation in flow
through a micro-orifice inside a silicon microchannel} from Chandan
Mishra, and Yoav Peles. This paper describes the manufacturing process
of glass bonded Si chips for high-pressure water flows. *
\href{pdf/microfluidic\%20venturi.pdf}{An experimental investigation of
hydrodynamic cavitation in micro-Venturis} from Chandan Mishra, and Yoav
Peles. This paper describes a venturi and the phase transition of the
liquid into vapor.

Work in two groups. The first group focuses reading and understanding on
manufacturing. Try to answer the questions: 1. Explain the use of a
photoresist. 2. What is the difference of SiO and pure Si surface to RIE
and DRIE processes? 3. Where is anodic bonding used. 4. How is the
device mounted in the holder and connected to the tubing?

The second group looks at the experimental paper (cavitation in
micro-Venturis). 1. The cavitation number and the Reynolds number are
defined. Check their values for the experimental protocols. Is it a low
or high-Reynolds number flow? What dominates, inerta or viscosity? 2.
Why does a low cavitation number result in more vapor generation? 3.
Summarize the regimes of cavitation observed in the experiment.

\subsection{Week 7}\label{week-7-1}

Derivation of the force on a sphere \href{Potential\%20Sphere.ipynb}{in
potential flow} and in a viscous flow, i.e. the
\href{Stokes\%20Sphere.ipynb}{Stokes Sphere} problem. We will also look
at a viscous microfluidic flow with a potential flow solution, the
\href{Hele\%20Shaw\%20Flow.ipynb}{Hele Shaw flow}.

\subsection{Tutorial 4 (Week 8)}\label{tutorial-4-week-8-1}

\subsubsection{Question 1}\label{question-1-2}

\begin{itemize}
\item
  Derive the Navier Stokes Solution to the steady flow of a film of
  liquid of uniform thickness \(h\) that flows done an incline (dark
  gray) under an angle \(\alpha\). Use the fact that there is no
  pressure gradient driving the flow (why?). Also set the \(x\)-axis
  parallel to the incline and choose a suitable body force.
\item
  Let the liquid be water and calculate the speed \(v_x(h)\) of the free
  surface of the film in the case \(h=100\,\mu\)m und angle
  \(\alpha=30^\circ\).
\end{itemize}

\subsubsection{Question 2}\label{question-2-2}

\begin{itemize}
\tightlist
\item
  Look up literature on the Milikan Experiment. Find out how the Stokes
  Law \(F_D=6\pi\mu R U\) was used to determine the charge of an
  electron. Sketch the forces acting on a droplet and explain who the
  experiment was conducted in detail.
\end{itemize}

\subsubsection{Question 3}\label{question-3}

\begin{itemize}
\tightlist
\item
  Aerosols are fine droplets in the air that may contain viral load. In
  the current pandemic researcher are trying to understand the transport
  of these droplets. In a quiestence room with no drift (net flow) of
  the air, droplets may fall by gravity. Calculate and plot the fall
  time of an aequous droplet from \(100\,\)nm to \(10\,\mu\) spherical
  droplet in air for a distance of \(30\,\)cm. Check and compare your
  findings with a recent
  \href{https://www.pnas.org/content/early/2020/05/12/2006874117}{paper}.
  Comment critically this paper.
\end{itemize}

\subsection{Week 8}\label{week-8}

\subsection{Week 9}\label{week-9}

\subsection{Week 10}\label{week-10}

\subsection{Week 11}\label{week-11}

\subsection{Week 12}\label{week-12}

\subsection{Week 13}\label{week-13}

\subsection{Week 14}\label{week-14}

\subsection{Week 8}\label{week-8-1}

\subsection{Week 9}\label{week-9-1}

\subsection{Week 10}\label{week-10-1}

\subsection{Week 11}\label{week-11-1}

\subsection{Week 12}\label{week-12-1}

\subsection{Week 13}\label{week-13-1}

\subsection{Week 14}\label{week-14-1}


    % Add a bibliography block to the postdoc
    
    
    
    \end{document}
